%%%%%%%%%%%%%%%%%%%%%%%%%%%%%%%%%%%%%%%%%
% Hand over doc
%
%%%%%%%%%%%%%%%%%%%%%%%%%%%%%%%%%%%%%%%%%

%----------------------------------------------------------------------------------------
%	PACKAGES AND OTHER DOCUMENT CONFIGURATIONS
%----------------------------------------------------------------------------------------

\documentclass[9.5pt]{article} % Default font size is 12pt, it can be changed here

\usepackage{geometry} % Required to change the page size to A4
\geometry{a4paper} % Set the page size to be A4 as opposed to the default US Letter

\usepackage{graphicx} % Required for including pictures

\usepackage{float} % Allows putting an [H] in \begin{figure} to specify the exact location of the figure
\usepackage{wrapfig} % Allows in-line images such as the example fish picture

\usepackage{lipsum} % Used for inserting dummy 'Lorem ipsum' text into the template

\usepackage{hyperref}

\linespread{1} % Line spacing

%\setlength\parindent{0pt} % Uncomment to remove all indentation from paragraphs

\begin{document}

%----------------------------------------------------------------------------------------
%	TITLE PAGE
%----------------------------------------------------------------------------------------

\begin{titlepage}

\newcommand{\HRule}{\rule{\linewidth}{0.5mm}} % Defines a new command for the horizontal lines, change thickness here

\center % Center everything on the page

\textsc{\Large Jesus College}\\[0.5cm] 
\textsc{\large Middle Combination Room}\\[0.5cm] 
\HRule \\[0.4cm]
{ \huge \bfseries Social officer handover document}\\[0.4cm] 
\HRule \\[1.5cm]

\begin{minipage}{0.8\textwidth}
\begin{flushleft} \large
\emph{2013 reps:}\\
James Black \emph{james.black@cantab.net}

Luke Sperrin \emph{luke.sperrin@gmail.com}


\emph{2012 rep:}\\
Tina Schwamb \emph{TinaSchwamb@gmail.com}
\end{flushleft}
\end{minipage}\\[4cm]
\includegraphics{Crest2}\\[1cm] 

{\large Last updated \today}\\[3cm] % Date, change the \today to a set date if you want to be precise

\vfill % Fill the rest of the page with whitespace

\end{titlepage}

%----------------------------------------------------------------------------------------
%	TABLE OF CONTENTS
%----------------------------------------------------------------------------------------

\tableofcontents % Include a table of contents

\newpage 

%----------------------------------------------------------------------------------------
%	INTRODUCTION
%----------------------------------------------------------------------------------------

\section{Duties}
These are pretty obvious. You plan the social events. You also act as the liason with the college for booking rooms. Apart from the international officer and the academic officer, if anyone else want's stuff from catering you'll be talking to catering to sort out the room booking and catering request. This is most evident in freshers fortnight. You guys sort out the schedule with the barman, manciple and JCSU. You then collate everyones catering requests (i.e. tea party wants x plates and a kettle, cheese and wine wants y food trays, and z knives) into one document, which you then discuss with the Manciple.
You also need to make sure all events are allocated to someone, and chase them up to make sure they have their events sorted.

%------------------------------------------------
\subsection{Speech in freshers week} % Sub-section
You have to talk for a couple of minutes about your role on college. Some won't know what a swap is, so you could also mention that Cambridge tradition.
%------------------------------------------------
\subsection{Future and ongoing projects} % Sub-section
As the social officer you can add any event you want to the ones mentioned in this document. We know the grads want more frequent smaller events that don't need much work on your side beyond promoting it on facebook or in the bulletin (like movie nights, BBQ's etc) - but be aware that as the social officer if you put on any event, no matter how casual or small, you need that party permit if you don't want a lecture from the porters. Best to take it off college with a restaurant meet, Jesus Green etc, if you don't organise it well in advance.
%------------------------------------------------
\subsection{Committee membership} % Sub-section
As you guys will put in quite a bit more time into your role most of the committtee, although most if it is fun stuff like attending all the events you organise, you're spared sitting on any committee.

\section{Stuff to do first} % Major section
The manciple and the barman are your two goto people in college. Make sure to talk to them both well before September.
There are a few things that it pays to get done really quickly. They are listed here, and in the ``Freshers fortnight" handover.
Don't forgot that every single event on college grounds should have a party permit (from tea parties to super hall bops). You apply for these via the Dean on Jnet.


%------------------------------------------------

\subsection{Fire induction} % Sub-section

During the bops you have to provide 2 fire wardens. I would suggest pushing hard to get the non-drinkers on the MCR to do this, as the fire wardens only job is to know where the fire exits are, make sure they aren't blocked, then stay sober for the night. Therefore ask the whole committee to attend the induction (sometime over summer), then keep chasing up the sober people. Contact the fire porter, Peter Thorpe (p.thorpe@jesus.cam.ac.uk) for details. This has to done before freshers week, so best to get it over quickly. 

%------------------------------------------------

\subsection{Liquor induction} % Sub-section

Talk to James the bar man (JTBM) about this. It's a 20 minute induction that at least two (including one social officer) should do. It's pretty easy, and is just common sense stuff like don't do promotions focused around getting black out, etc. Again, must be done as soon as possible when JTBM gets back, and before you give out wine during the wine and cheese in Freshers Week, at the latest.


%------------------------------------------------

\subsection{Pre-dinner drinks} % Sub-section

This is actually the VP's responsibility, but as the social officer you liase with Lisa (the manciple) and so you are the one that will get given, or needs to ask for, the list of halls for the year. Once you get a list of when all the halls are, pass it on to the VP to organise the pre-dinner drinks.

%------------------------------------------------

%\subsubsection{Subsubsection 1} % Sub-sub-section

\subsection{MCR calendar}

If you go to \url{https://www.google.com/calendar/} and login using the MCR social email address (sign in details below). Any changes or events added to the ``MCR Social" calendar on this account will appear on the site (which gets about 1000 a visits a month) and will also be automatically be added to the calendars on the computers and phones of grads that have subscribed to the calendar. \emph{It is very important to keep this up to date!}
\begin{verbatim}
login: mcr-social@jesus.cam.ac.uk
Password: click lost password to set reset it via email
\end{verbatim}

%------------------------------------------------

%\subsubsection{Subsubsection 1} % Sub-sub-section

\subsection{Term card}

Up till 2011 a term card (list of major events) was printed and put in peoples pigeon holes. We moved to just the online calendar and updates in the bulletin in 2012. We did have a physical term card, but only a few were printed. Rather than printing one for every grad just a few were left in the grad room and a few given to Sheena. 

Sheena has stated that she prefers a printed version. At least a few grads also miss it. But quite a few printed term cards never leave the grad room, and it costs us money if you can't get free printing in your department, so whether to print 300ish term cards 3 times a year or not still needs to be sorted.......






%----------------------------------------------------------------------------------------
%	MAJOR SECTION X - TEMPLATE - UNCOMMENT AND FILL IN
%----------------------------------------------------------------------------------------

\section{Super halls}

There are 4 of these during the year – Halloween, Christmas, Burns Night and End of Year Dinner. For each, meet with Catering at least 2 months in advance to organise the menus, make room bookings etc.
							
For Halloween, Xmas and End of Year Super-Hall you will need to organise a bop in the bar afterwards – arrange this with the barman and Catering. You can get the DJ equipment from the JCSU (email their ents officer beforehand). If the bop is out of term you might be able to extend it by half an hour (drinks until midnight, music until 12:30) – ask the dean in the party permit, but do not forget to ask Lisa and the Barman.
					
A super-hall is a lot of work and you need to involve your committee. It's easiest just to dole out jobs to committee members. Tasks included: Decorate/prepare bar before bop (remove chairs/tables), Serve pre-dinner drinks, Fire warden, Tidy bar after bop etc. Make everyone do one thing at least.

\emph{The big debate over superhalls is do we max out numbers, and have pre-drinks in the forum, or ask Lisa to limit ticket sales to the max that will fit in the bar and coleridge for the bop.} Also make sure to sort out guests - you have to ask for one guest each time. We brought in the 2 days of no guests, then one guest allowed later. This was due to super halls selling out in 6 hours, and due to the high number of people that sold on their guest tickets, so one person could bring multiple guests (one person requested a table with the guests of 6 other people, so the on selling of spare guests tickets is quite prolific). This worked in increasing the number of Jesus grads that get tickets, but can also make partners that live in college feel a bit shit \emph{I (James) had one complaint about the no-guests for the first 2 days, but that was from someone wanting to bring multiple guests, which is exactly what we were wanting to prevent.} Best to discuss with committee again. Ultimately the ticket policy can be whatever the MCR president signs off on. 

We also used any money left in each super halls budgets on prizes. Sheena was our go to judge (as a thanks for her support).



\subsection{Halloween} % Sub-section
In 2011 Tina had a pumpkin carving competition within MCR to help us carve the pumpkins. However, only 3 people showed up for that one. We cancelled the event in  2012, which proved sufficient to motivate enough people to want it. In the end numbers were quite good. Pumpkins were provided by catering and had to be picked up beforehand.
							
Usually JSCU organise their Halloween formal the same week and, it may be best to share the pumpkins with JCSU to avoid food wastage.							
We also ran a best-costume competition, and at the end of the dinner awarded a free Grad Hall ticket to the best dressed.
							
You should also buy Halloween decorations for the bar. I spend £20 pounds on candy and junk, and spread it on the tables before dinner.

\subsection{Xmas} % Sub-section
A black tie superhall. We had a brass band play from the balcony before dinner, and didn't have pre-drinks (for this hall we went for a full upstairs, so there was nowhere in college big enough for pre-drinks except the forum, which we didn't want). We did mulled cider, which was about 25\% the price of mulled wine. And tasted pretty good. You need to drop off the booze and spices off with the cellar man, and the kitchen makes the mulled cider for you.

\subsection{Burn's night} % Sub-section
You need to find someone to pipe the haggis into hall, try Angus Fitchie (He was a Jesus student till 2011, not sure if he’s still around: af369@cam.ac.uk).
					
You have to organise speeches during dinner. We had 5 speeches: The Selkirk Grace (very short), Address to a haggis, Immortal memory speech*, Toast to the Lassies*, Reply to the Toast*. Try to find Scottish grads to give these speeches. You can offer them a free ticket.
* = More work as they have to come up with the speech themselves
Make sure to tell them the toast to the lassies and the reply are tounge and cheek speeches. The men are thanking the woman for the food, and the woman can be rude in reply.
					
Just before the Haggis speech everyone gets a glass of whiskey. You have to provide the whiskey and catering serves it. Also ask the MCR president to invite the Scottish fellows, the grad tutors and chaplain.
					
Usually there’s a ceilidh after dinner in the forum. Contact the University Ceilidh band (secretary@cucb.co.uk) early to make sure they are free. I think they charged us \pounds{300}. Contact the JCSU so that you can use the forum.

\subsection{End of year dinner} % Sub-section

This super-hall starts earlier at 6pm with pre-drinks. They are provided by catering and are included in the ticket prize. If the weather is nice, they are served in 2nd court, otherwise in the graduation marquee on the cricket pitch. This year and the year before we had some Jazz music with it. You can ask Jesus Jazz band (we paid \pounds{150} for 4 people and 1 hour of Jazz, but they are flexible if you have not enough budget). They needed electricity for the instruments – you can speak to maintenance about this.
					
You can make up a theme; this year we had 1920s; the year before it was masquerade.
					
It’s also nice to inform the Master, Master’s wife, the grad tutors, nurse Poskitt, the manciple and grad secretary beforehand and officially invite them (before tickets go on sale, so you can reserve their tickets). The MCR doesn’t pay for them, they just need to tell catering that they want to attend.
					
There’s usually an end-of-year photograph taken in hall before dinner. We got a grad to do it, which meant we could give every grad a digital copy of the photo. A tripod is essential as the light isn't great- we used a mid-range (non-SLR) camera in 2013. I would strongly recommend finding an SLR camera in future years.
					
Again, there’s a bop in the bar afterwards. As decoration I put pictures from the whole year up. Talk to JTBM to get the bar, before telling Lisa.


%----------------------------------------------------------------------------------------
%	MAJOR SECTION 1
%----------------------------------------------------------------------------------------

The events listed below are just the main ones. As social officer, it's up to you to continue to improve the current stock as you see fit, which to retire, and which new ones you wish to introduce. 

\section{Other events}
\subsection{Swaps}
Swaps are an essential part of the Cambridge experience. Go after the big colleges early. It also pays to keep the special sounding jesus halls, like St Paddies, to help you entice the big Colleges like King's or Johns into locking in a swap early. I aimed for 15 swaps in 2012/13, but about 10 a year is normal.
\subsubsection{Colleges}
Be aware that Jesus grads are picky. Kings, Johns, Trinity, Kings, Peterhouse etc will be well attended and sell out quickly. 
Colleges like Queens, were slower to sell. As an example, we had 20+ people apply for King's in a few hours, while Selwyn had just 10 sign up over 2 days. While we haven't done swaps with colleges like Lucy, Hughes, Robinson, Homerton, etc, in recent years, interest from grads seems pretty low. 
I usually chased after the old colleges, and if someone like Queens or Girton approached, I would ask a few grads to get a feel for interest.
\subsubsection{When they visit}
Talk to the manciple to buy 10 tickets. Find out from the treasurer how he/she wants to be paid. In our year that was by a cheque, made out to ``jesus college MCR". On the night, met them at the Plodge, take them to pre-drinks, and then up to the hall. It's nice to go to the hall with them, but not that necessary. For the one or two swaps Luke or I didn't go to, I just asked someone I knew to go over after dinner and offer to show them where the bar was.

\subsubsection{Selling tickets}
I (James) set up a wordpress site in 2012 to sell tickets, and automate the whole process. I had timers set up, and so all I had to do was add a new swap, say when I wanted it to open, how many tickets I wanted to sell, and then it would automatically open at the right time and collect all the info needed. I used a google form, which I parsed into the site, so it had my own styling. This automatically wrote the results to a google spreadsheet like a normal google form. It worked great, and if needed I can give login details. Otherwise you can just do it manually, using a google form. I had an email asking me not to open swaps between 8am and 10pm, due to some people being in labs without internet access. 

\subsubsection{Getting the money}
We used gocardless to collect payments in my year. It made life much easier for me and the tres as it keeps track of payments for you. Just make a paylink, and email it out. They click on it, and the MCR gets paid (so you never have to handle money, and both the person paying and the treasurer has a receipt for who has paid). Those without UK bank accounts still paid by cash (we had one person in 2012/13 that couldn't use gocardless), by putting the money in the tres' pigeon hole.

And important note is that this a direct debit system. So in the case of fraud, the students are protected by RBS, but after RBS refunds them they will most likely investigate any fraud. Not really an issue here, as we only do one-off payments, so we can't take more than we are suppose to from peoples accounts, but all the more reason to be very careful with the password (I collected \pounds{1600}+ plus for swaps using gocardless in 2012/13, so it's a fair bit of cash).
\subsection{Grad bar night}
We started this in 2013. Talk to JTBM, to see when in the holidays he can do it (so the undergrads are not around). We brought in the xbox, games and stuff from the grad room. And invited Clare and Emma. Turn out was poor till about 9pm, then it was packed out by closing. It would be cool to find away to get them to come in casually and hang out earlier in the night, maybe with free pizza or something, but is worth repeating as there were around 60 grads for the last hour and a half.


\subsection{MCR bake off}
This event was the idea of a grad (Goylette Chami, gjc36@cam.ac.uk). Turn out was great. Basically we invited to grads to bake stuff, and had a competition in the grad room. We gave three tickets to halls away to the three best. Could be good to get someone like the Chaplain, the grad tutors etc to be the judges. Hardly any work for the social officer. Just get the party permit, advertise it, and set up the tables on the day.
\subsection{Wine tasting}
We did this with Caius. No work for us, we just had to sell tickets. Hardly any work as the Cambridge Wine Merchants will run it for you if Caius don't invite us again. 
\subsection{Beer tasting}
We did this at the Cambridge Brewhouse (on Malcolm St). The brewmaster, James, is a good dude, but useless at emails. I found the only way to get stuff sorted was to go down and see him. I would suggest talking to Clare college early, and inviting them as well. They can do 40+ for a tasting. I asked Clare for a ``tasting swap", where we organised the beer tasting, they did the whiskey. I'd recommend collecting money before the event, as we had one person disappear when it came time to pay in 2013 (a friend covered him though).
\subsection{Whisky night}
Clare's whisky night is by far the ticket to get in on. They have their own whisky bar, and 40 whisky snifters, so are bloody well set up for it. I used the beer tasting as a bargaining chip to get in with them. If Clare don't invite us, one of the old Clare barmen set up the University Whisky society (\href{mailto:whiskysoc@cusu.cam.ac.uk}{whiskysoc@cusu.cam.ac.uk})
\subsection{Speed dating}
This was done by a grad (Steph) this year, to raise funds for charity. We did it with the City Rowing club. I got a complaint about it being with townies - which probably implies we should have more events like this with people outside the uni.





%----------------------------------------------------------------------------------------

\end{document}